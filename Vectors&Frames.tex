\documentclass{article}
\usepackage{tikz}
\usepackage{amsmath, amssymb, amsthm}
% Customization ---------------

% GEOMETERY
% sets paper size, margins and a parameter 
\usepackage[letterpaper, top=1.3in, bottom=1.3in, left=1.3in, right=1.3in,heightrounded]{geometry}

%Background
\usepackage{background}
\backgroundsetup{
 scale=1,
 color=black,
 opacity=10,
 angle=0,
 contents={
  \includegraphics[width=\paperwidth,height=\paperheight]{brand.pdf}
 }
}
%Line Height 
\renewcommand{\baselinestretch}{1.25} %Line spacing

%Par Skip and parindent
\setlength{\parindent}{0pt}
\setlength{\parskip} {1.2em}
\usetikzlibrary{arrows}

\newcommand{\vect}[1]{\langle #1 \rangle}
\begin{document}


\title{\bf CSCI 340, Winter 2025\\ Math HW01}
\author{\textbf{Name}: \textit{Abid Jeem}}
\date{}

\maketitle


Show your work.  Explain, in each case, how you got a particular
numerical result.  You may, of course, use a computer for numeric
computations, but which expressions you used to get each result must
be made clear.  For example, you can't just put down 12, you must put
down $2^2 + 2^2 + 2^2 = 12$.  You can't just put down 4.443, you
must put down $\pi\cdot\sqrt{2} = 4.443$

If you have trouble writing neat math, consider learning and using \LaTeX.

\begin{enumerate}
\item Let $v=\vect{2,2,1}$ and $w=\vect{1,-2,0}$.  Find the following:
  \begin{enumerate}
  \item $v\cdot w$
  
\textbf{Answer:} Computing dot product we get — 
\begin {equation}
(2*1) + (2*-2) + (1*0) 
= \quad \boxed{-2} 
\end {equation}

  \item The vector projection of $w$ on $v$.

\vspace{1em}
\textbf{Answer:} Computing projection formula—
\begin {equation}
    \left(\frac{w \cdot v}{v \cdot v}\right) v
    = \left(\frac{-2}{2^2 + 2^2 + 1^2}\right) \vect{2,2,1}
    = \left(\frac{-2}{9}\right)\vect{2,2,1}
    = \quad \boxed{\vect{-\tfrac{4}{9},\, -\tfrac{4}{9},\, -\tfrac{2}{9}}}
\end {equation}

 \end{enumerate}

\item Normalize each of the following vectors, in other words, make
  each vector unit length.
  
  \begin{enumerate}
  \item $v_1 = \vect{\frac{\sqrt{2}}{2}, \frac{\sqrt{2}}{2}, 0}$ 
  
\vspace{1em}
\textbf{Answer: }

\begin {equation}
    \frac{v_1}{\|v_1\|}
    = \frac{\vect{\frac{\sqrt{2}}{2}, \frac{\sqrt{2}}{2}, 0}}{\sqrt{\left(\frac{\sqrt2}{2}\right)^2 + \left(\frac{\sqrt2}{2}\right)^2 + 0^2}}
  = \frac{\vect{\frac{\sqrt{2}}{2}, \frac{\sqrt{2}}{2}, 0}}{1}
  =\quad \boxed{{\vect{\frac{\sqrt{2}}{2}, \frac{\sqrt{2}}{2}, 0}}}
\end {equation}

\vspace{1em}

  \item $v_2 = \vect{-1,1,-1}$

\vspace{1em}
  \textbf{Answer: }
  
\begin {equation}
    \frac{v_2}{\|v_2\|}
    = \frac{\vect{-1,1,-1}}{\sqrt{(-1)^2 + (1)^2 +(-1)^2 }}
  = \frac{\vect{-1,1,-1}}{\sqrt{3}}
  =\quad \boxed{\vect{-\frac{1}{\sqrt{3}}, \frac{1}{\sqrt{3}}, -\frac{1}{\sqrt{3}}}}
\end {equation}

  \item $v_3 = \vect{0,-2,-2}$

\vspace{1em}
\textbf{Answer: }

\begin{equation}
    \frac{v_3}{\|v_3\|}
    = \frac{\vect{0, -2, -2}}{\sqrt{(0)^2 + (-2)^2 + (-2)^2}}
    = \frac{\vect{0, -2, -2}}{\sqrt{8}}
    = \quad \boxed{\vect{0, -\frac{2}{\sqrt{8}}, -\frac{2}{\sqrt{8}}}}
\end{equation}

  \end{enumerate}

\item Find the cosine of the angle between the vectors
  $v=\vect{1,2,3}$ and $w=\vect{3,2,1}$
  
\vspace{1em}
\textbf{Answer: }

Here, from the dot product equation

\begin{equation}
\cos(\theta) = \frac{v \cdot w}{\|v\| \|w\|} 
= \frac{(1)(3) + (2)(2) + (3)(1)}{\sqrt{(1)^2 + (2)^2 + (3)^2} \sqrt{(3)^2 + (2)^2 + (1)^2}}
= \frac{10}{\sqrt{14} \sqrt{14}} 
= \frac{10}{14} 
= \quad \boxed{\frac{5}{7}}
\end{equation}



    \newpage
  \item 
  
    From the diagram, we get—
\begin{align*}
p &= (3, 7) \\
q &= (9, 3) \\
u &= (4 - 2, 1 - 1) = (2, 0) \\
v &= (9 - 5, 5 - 9) = (4, -4) \\
w &= (7 - 5, 5 - 3) = (2, 2)
\end{align*}
    \begin{enumerate}
      \item The coordinates of $q$ in the f = $\langle p, u, v \rangle$:
        \[
        q = p + a u + b v \quad \Rightarrow \quad 
        \begin{aligned}
            &q_x = p_x + a u_x + b v_x  \\
            &q_y = p_y + a u_y + b v_y
        \end{aligned}
        \]

        Substituting the values:
\[
\begin{aligned}
    &q_x = 9, \, p_x = 3, \, u_x = 2, \, v_x = 4 \\
    &q_y = 3, \, p_y = 7, \, u_y = 0, \, v_y = -4
\end{aligned}
\]

Step-by-step calculations:
\[
\begin{aligned}
    &q_x = p_x + a u_x + b v_x \quad \Rightarrow \quad 9 = 3 + 2a + 4b \quad \Rightarrow \quad 6 = 2a + 4b \quad \Rightarrow \quad a + 2b = 3 \\
    &q_y = p_y + a u_y + b v_y \quad \Rightarrow \quad 3 = 7 + 0a - 4b \quad \Rightarrow \quad -4 = -4b \quad \Rightarrow \quad b = 1
\end{aligned}
\]

Substituting $b = 1$ into $a + 2b = 3$:
\[
a = 3 - 2(1) = 1
\]

Final result:
\[
\quad \boxed{q = (1, 1)_{\langle p, u, v \rangle}}
\]

        
        
      \item The coordinates of $q$ in the frame $\langle p, u, w\rangle$
\[
        q = p + a u + b w \quad \Rightarrow \quad 
        \begin{aligned}
            &q_x = p_x + a u_x + b w_x \\
            &q_y = p_y + a u_y + b w_y
        \end{aligned}
        \]

        Substituting the values:
\[
\begin{aligned}
    &q_x = 9, \, p_x = 3, \, u_x = 2, \, w_x = 2 \\
    &q_y = 3, \, p_y = 7, \, u_y = 0, \, w_y = 2
\end{aligned}
\]

Step-by-step calculations:
\[
\begin{aligned}
    &q_x = p_x + a u_x + b w_x \quad \Rightarrow \quad 9 = 3 + 2a + 2b \quad \Rightarrow \quad 6 = 2a + 2b \quad \Rightarrow \quad a + b = 3 \\
    &q_y = p_y + a u_y + b w_y \quad \Rightarrow \quad 3 = 7 + 0a + 2b \quad \Rightarrow \quad -4 = 2b \quad \Rightarrow \quad b = -2
\end{aligned}
\]

Substituting $b = -2$ into $a + b = 3$:
\[
a = 3 - (-2) = 3 + 2 = 5
\]

Final result:
\[
\quad \boxed{q = (5, -2)_{\langle p, u, w \rangle}}
\]
      \item The coordinates of $q$ in the frame $\langle p, v, w\rangle$
\[
        q = p + a v + b w \quad \Rightarrow \quad 
        \begin{aligned}
            &q_x = p_x + a v_x + b w_x \\
            &q_y = p_y + a v_y + b w_y
        \end{aligned}
        \]
        Substituting the values:
\[
\begin{aligned}
    &q_x = 9, \, p_x = 3, \, v_x = 4, \, w_x = 2\\
    &q_y = 3, \, p_y = 7, \, v_y = -4, \, w_y = 2
\end{aligned}
\]

Step-by-step calculations:
\[
\begin{aligned}
    &q_x = p_x + a v_x + b w_x \quad \Rightarrow \quad 9 = 3 + 4a + 2b \quad \Rightarrow \quad 6 = 4a + 2b \quad \Rightarrow \quad 2a + b = 3 \\
    &q_y = p_y + a v_y + b w_y \quad \Rightarrow \quad 3 = 7 - 4a + 2b \quad \Rightarrow \quad -4 = -4a + 2b \quad \Rightarrow \quad 2a - b = 2
\end{aligned}
\]

Adding the equations:
\[
(2a + b) + (2a - b) = 3 + 2 \quad \Rightarrow \quad 4a = 5 \quad \Rightarrow \quad a = \frac{5}{4}
\]

Substituting $a = \frac{5}{4}$ into $2a + b = 3$:
\[
2\left(\frac{5}{4}\right) + b = 3 \quad \Rightarrow \quad \frac{10}{4} + b = 3 \quad \Rightarrow \quad b = 3 - \frac{10}{4} = \frac{12}{4} - \frac{10}{4} = \frac{2}{4} = \frac{1}{2}
\]

Final result:
\[
\quad \boxed{q = \left(\frac{5}{4}, \frac{1}{2}\right)_{\langle p, v, w \rangle}}
\]

      \item The coordinates of $p$ in the frame $\langle q, u, v\rangle$
\[
        p = q + a u + b v \quad \Rightarrow \quad 
        \begin{aligned}
            &p_x = q_x + a u_x + b v_x \\
            &p_y = q_y + a u_y + b v_y
        \end{aligned}
        \]
        Substituting the values:
\[
\begin{aligned}
    &p_x = 3, \, q_x = 9, \, u_x = 2, \, v_x = 4 \\
    &p_y = 7, \, q_y = 3, \, u_y = 0, \, v_y = -4
\end{aligned}
\]

Step-by-step calculations:
\[
\begin{aligned}
    &p_x = q_x + a u_x + b v_x \quad \Rightarrow \quad 3 = 9 + 2a + 4b \quad \Rightarrow \quad -6 = 2a + 4b \quad \Rightarrow \quad a + 2b = -3 \\
    &p_y = q_y + a u_y + b v_y \quad \Rightarrow \quad 7 = 3 + 0a - 4b \quad \Rightarrow \quad 4 = -4b \quad \Rightarrow \quad b = -1
\end{aligned}
\]

Substituting $b = -1$ into $a + 2b = -3$:
\[
a + 2(-1) = -3 \quad \Rightarrow \quad a - 2 = -3 \quad \Rightarrow \quad a = -1
\]

Final result:
\[
\quad \boxed{p = (-1, -1)_{\langle q, u, v \rangle}}
\]

      \item The coordinates of $p$ in the frame $\langle q, u, w\rangle$
\[
        p = q + a u + b w \quad \Rightarrow \quad 
        \begin{aligned}
            &p_x = q_x + a u_x + b w_x \\
            &p_y = q_y + a u_y + b w_y
        \end{aligned}
        \]

        Substituting the values:
\[
\begin{aligned}
    &p_x = 3, \, q_x = 9, \, u_x = 2, \, w_x = 2 \\
    &p_y = 7, \, q_y = 3, \, u_y = 0, \, w_y = 2
\end{aligned}
\]

Step-by-step calculations:
\[
\begin{aligned}
    &p_x = q_x + a u_x + b w_x \quad \Rightarrow \quad 3 = 9 + 2a + 2b \quad \Rightarrow \quad -6 = 2a + 2b \quad \Rightarrow \quad a + b = -3 \\
    &p_y = q_y + a u_y + b w_y \quad \Rightarrow \quad 7 = 3 + 0a + 2b \quad \Rightarrow \quad 4 = 2b \quad \Rightarrow \quad b = 2
\end{aligned}
\]

Substituting $b = 2$ into $a + b = -3$:
\[
a + 2 = -3 \quad \Rightarrow \quad a = -3 - 2 \quad \Rightarrow \quad a = -5
\]

Final result:
\[
\quad \boxed{p = (-5, 2)_{\langle q, u, w \rangle}}
\]

      \item The coordinates of $p$ in the frame $\langle q, v, w\rangle$
\[
        p = q + a v + b w \quad \Rightarrow \quad 
        \begin{aligned}
            &p_x = q_x + a v_x + b w_x \\
            &p_y = q_y + a v_y + b w_y
        \end{aligned}
        \]

        Substituting the values:
\[
\begin{aligned}
    &p_x = 3, \, q_x = 9, \, v_x = 4, \, w_x = 2 \\
    &p_y = 7, \, q_y = 3, \, v_y = -4, \, w_y = 2
\end{aligned}
\]

Step-by-step calculations:
\[
\begin{aligned}
    &p_x = q_x + a v_x + b w_x \quad \Rightarrow \quad 3 = 9 + 4a + 2b \quad \Rightarrow \quad -6 = 4a + 2b \quad \Rightarrow \quad 2a + b = -3 \\
    &p_y = q_y + a v_y + b w_y \quad \Rightarrow \quad 7 = 3 - 4a + 2b \quad \Rightarrow \quad 4 = -4a + 2b \quad \Rightarrow \quad -2a + b = 2
\end{aligned}
\]

Adding the equations:
\[
(2a + b) + (-2a + b) = -3 + 2 \quad \Rightarrow \quad 2b = -1 \quad \Rightarrow \quad b = -\frac{1}{2}
\]

Substituting $b = -\frac{1}{2}$ into $2a + b = -3$:
\[
2a - \frac{1}{2} = -3 \quad \Rightarrow \quad 2a = -3 + \frac{1}{2} \quad \Rightarrow \quad 2a = -\frac{6}{2} + \frac{1}{2} = -\frac{5}{2} \quad \Rightarrow \quad a = -\frac{5}{4}
\]

Final result:
\[
\quad \boxed{p = \left(-\frac{5}{4}, -\frac{1}{2}\right)_{\langle q, v, w \rangle}}
\]

    \end{enumerate}
\newpage
    \tikzset{>=latex}
    \begin{tikzpicture}
      \draw[dotted] (0,0) grid (9.9,9.9);
      \draw[thick,<->] (10,0) node[anchor=north west] {$x$ axis}
      -- (0,0) -- (0,10) node[anchor=south east] {$y$ axis};
      \foreach \x in {0,1,2,3,4,5,6,7,8,9}
        \draw (\x cm,2pt) -- (\x cm,-2pt) node[anchor=north] {$\x$};
      \foreach \y in {0,1,2,3,4,5,6,7,8,9}
        \draw (2pt,\y) -- (-2pt,\y cm) node[anchor=east] {$\y$};

      \fill (3,7) circle (3pt) node[anchor=north west] {$p$};
      \fill (9,3) circle (3pt) node[anchor=north west] {$q$};

      \draw[->,thick] (2,1) -- (4,1) node[midway,anchor=south] {$u$};
      \draw[->,thick] (5,3) -- (7,5) node[midway,anchor=south] {$w$};
      \draw[->,thick] (5,9) -- (9,5) node[midway,anchor=south] {$v$};

    \end{tikzpicture}

\end{enumerate}
\end{document}
